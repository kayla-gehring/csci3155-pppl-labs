\documentclass[a4paper, 10pt]{article}
\usepackage{fullpage} % changes the margin
\usepackage{enumitem}
\usepackage{algorithm}
\usepackage{listings}
\lstset
{
  language=Scala,
  basicstyle=\footnotesize,
  numbers=left,
  stepnumber=1,
  showstringspaces=false,
  tabsize=1,
  breaklines=true,
  breakatwhitespace=false,
}

\begin{document}
\noindent
\large\textbf{Lab 1} \hfill \textbf{Kayla Gehring} \\
\normalsize CSCI 3155, Spring 2018 \hfill Teammate: Tom Eldar \\

\section*{Written Problems}

\subsection*{2. Scala Basics: Binding and Scope}
\textit{For each of the following uses of names, give the where that name is bound. Briefly explain your reasoning.}

\begin{enumerate}[label=(\alph*)]

\item\textit{The use of pi at line 4 is bound at which line? The use of pi at line 7 is bound at which line?}

  \begin{lstlisting}[language=Scala]
    val pi = 3.14
    def circumference(r: Double): Double = {
      val pi = 3.14159
      2.0 * pi * r
    }
    def area(r: Double): Double =
      pi * r * r
  \end{lstlisting}

  \vspace{0.2in}

    \large{Line 4 is within the definition of `circumference' from lines 2-5, and uses the value 'pi' bound at line 3 (val pi = 3.14159).

    Line 7 is within the definition of 'area,' which does not bind 'pi' to any value within it, and therefore uses the value 'pi' bound in the global scope at line 1 (val pi = 3.14).}

    \vspace{0.2in}

  \item
    \textit{The use of x at line 3 is bound at which line? The use of x at line 6 is bound at which line? The use of x at line 10 is bound at which line? The use of x at line 13 is bound at which line?}

    \begin{lstlisting}[language=Scala]
      val x = 3
      def f(x: Int): Int =
      x match {
        case 0 => 0
        case x => {
          val y = x + 1
          ({
            val x = y + 1
            y
          } * f(x - 1))
        }
      }
    val y = x + f(x)
  \end{lstlisting}

  \vspace{0.2in}

  \large{The use of x at line 3 is bound at line 2 (def f(x: Int): Int = ), because it's within the scope of the function 'f'.

  The use of x at line 6 is bound at line 2, again because it's within the scope of function 'f,' and previous uses of 'x' in the scope are for pattern matching.

  The use of x at line 10 is bound at line 2. The binding of x at line 8 is only valid at line 8 and 9, and ends at the \{ on line 10.

  The use of x at line 13 is bound at line 1 (val x = 3) in the global scope. This is because line 13 outside the scope of the function 'f' and all scopes within 'f'.}

  \vspace{0.2in}

\end{enumerate}

  \subsection*{3. Scala Basics: Typing}
  \textit{In the following. I have left off the return type of function g. The body of g is well-typed if we can come up with a valid return type. Is the body of g well-typed?}
  \begin{lstlisting}[language=Scala]
    def g(x: Int) = {
      val (a, b) = (1, (x, 3))
      if (x == 0) (b, 1) else (b, a + 2)
    }
  \end{lstlisting}

  \vspace{0.2in}
  Yes, the body expression of 'g' is well-typed with a nested tuple of type (Int,(Int, Int)).

\begin{verbatim}
(b, c):((Int, Int), Int) because
     b:(Int, Int) because
          val (a, b) = (1, (x, 3))
               x:Int
               3:Int
     c:Int because
          1:Int (when x == 0)
          a + 2:Int (when x != 0) because
               val (a, b) = (1, (x, 3))
                   1:Int
               2:Int
\end{verbatim}

  \vspace{0.2in}

\end{document}
